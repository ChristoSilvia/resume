% Latex file for resume.  Requires `res.cls`
\documentclass[12pt]{res}

\usepackage{extsizes}
\usepackage{geometry}
\usepackage{hyperref}
\usepackage[utf8]{inputenc}

\geometry{
	letterpaper,
	left=0.5in,
	textwidth=6.75in,
	top=0.5in,
}

\begin{document}
% [12pt] adds a blank line after the name.
\begin{centering}
{\bf CHRISTOPHER P. SILVIA}\\
(240)-515-5838\\
\href{mailto:cps232@cornell.edu}{cps232@cornell.edu}\\
\end{centering}

\vspace{-12pt}

\address{\bf Present Address \\608 East Buffalo St.\\Ithaca, NY 14850}
\address{\bf Permanant Address \\ 7801 Leesburg Drive \\ Bethesda MD, 20817}


\begin{resume}
  \section{Education}

	\noindent
    {\bf Cornell University, Ithaca, NY}\\
    Bachelor of Arts, Physics with concentration in Applied Mathematics\\
    Bachelor of Arts, Mathematics, with concentration in Mathematical Physics\\
    Expected Graduation Date: May 2017 \\
    GPA: 3.64/4.0

  \vspace{-12pt}
  \section{Relevant Coursework}
  \vspace{-10pt}
    \begin{tabbing}
    \hspace{3.0in} \= \kill
    {\bf Undergraduate Courses } \> {\bf Graduate Courses} \\
    Fluid Dynamics		\>	Computational Fluid Dynamics\\
    Robotic Manipulation	\>	Mobile Sensor Planning\\
    Computational Physics	\>	Multivariable Control Theory\\
    \end{tabbing}


  \vspace{-12pt}
  \section{Honors and Awards}
	\noindent
    Dean's List of Distinguished Students: Fall 2013, Spring 2014, Fall 2014, Spring 2015, Fall 2015\\


  \vspace{-10pt}
  \section{Relevant Experience}
   
    \vspace{-5pt} 
    \begin{tabbing}
      \hspace{2.3in} \= \hspace{2.6in} \= \kill
      {\bf Research Team Member} \> Professor Alexander Vladimirsky  \> Summer 2016 \\
		\> Department of Mathematics\\
		\> Cornell University \\
    \end{tabbing}

	\vspace{-45pt}
	\begin{itemize}
	\item Wrote path-planning algorithm to improve ambulance response times.
	\vspace{-10pt}
	\item Implemented in Python using NetworkX
	\vspace{-10pt}
	\item Implemented a custom Heap data structure
	\end{itemize}
    
    \vspace{5pt}
    \begin{tabbing}
      \hspace{2.3in} \= \hspace{2.6in} \= \kill
      {\bf Research Team Member} \> Professor Bob Strichartz  \> Summer 2015 \\
		\> Department of Mathematics\\
		\> Cornell University \\
    \end{tabbing}\vspace{-30pt}	

	\noindent
	I investigated properties of solutions to the Schrödinger equation on
		the Serpinski Gasket, a fractal space.
	I numerically solved the differential equation through discretization
		with a novel scheme, and equalled the previous theoretical bound
		on precision solutions for differential equations of this type.

	\vspace{5pt}
	\begin{tabbing}
		\hspace{2.3in} \= \hspace{2.6in} \= \kill
		{\bf Research Team Member} \> Professor John Phillip \> Summer 2014 \\
		\> Department of Physics\\
		\> Catholic University of America \\
	\end{tabbing} \vspace{-30pt}

	\noindent
	I investigated the dynamics of micromagentic systems.
	I used and wrote programs to simulate micromagnetic dynamics and produce
		theoretical predictions about the geometry-dependend properties
		of small ferromagnetic materials. 

    % Lockheed
    \begin{tabbing}
    \hspace{2.3in}\= \hspace{2.6in}\= \kill % setup two tab positions
    \textbf{ Intern } \> Lockheed Martin Corporation \> Summer 2013\\
                  \> Gaithersburg, MD
    \end{tabbing}\vspace{-10pt}

	\noindent
	I created a system to deploy servers automatically to execute
    hadoop and storm queries on a large dataset of healthcare
    records, as well as automated other systems administration tasks.

    % NIST
    \begin{tabbing}
    \hspace{2.3in}\= \hspace{2.6in}\= \kill % setup two tab positions
    \textbf{Electronics Aide} \> National Institute of Standards \> Summer 2012\\
						\> and Technology (NIST) \\
                           \> Gaithersburg, MD    
    \end{tabbing}\vspace{-10pt}
	\noindent
      I created a console to display active jobs and broken nodes
      on a computer cluster at NIST and repaired broken computers.

  \section{Skills}
	\noindent
	\begin{itemize}
	\item Creating and analyzing numerical models, which used both 
		linear algebra and differential equations to predict the behavior of a system, 
		and experience writing executive summaries \LaTeX to summarize my findings to a nontechnical audience.
	\item Writing robot control systems with Linux, Robot Operating System
		(ROS) using Python (numpy/scipy).
	\item Designing mechanical systems on a FIRST robotics
		competition robot using Autodesk Inventor, 
		as well as acting as lead engineer.
	\item Working on teams to deliver projects and write-ups
		in engineering and mathematical modeling classes,
		some involving collaborating on a significant amount of shared
		code using Git.
	\end{itemize}

	\section{Programming Languages}
	\noindent
	\begin{itemize}
	\item Proficient in Python, as well as its numpy and scipy libraries
	\item Experienced with Matlab, Julia, and Octave
	\end{itemize}

  \section{Relevant Coursework}
    \vspace{-5pt}
    \begin{tabbing}
      \hspace{2.3in} \= \hspace{2.3in} \= \kill
      {\bf Math}         \> {\bf Physics}      \> {\bf Engineering} \\
      Numerical Analysis \> Fluid Dynamics \> Mobile Sensor Planning \\
      Nonlinear Dynamics \> Thermodynamics \> Robotic Manipulation\\ 
      Mathematical Modeling \> Computational Physics \> Circuits  \\
      Complex Analysis \> Quantum Mechanics \> Lasers and Photonics\\
    \end{tabbing}

\end{resume}
\end{document}
